\subsubsection{Results and Analysis}
\paragraph{Metrics.}
For the centrality algorithms, we focus on \emph{wall-clock time} as the primary empirical metric. All timings were measured in seconds using Python's \texttt{time.perf\_counter} on the same machine and environment. All four implementations compute exact values (not approximations).

\paragraph{Scaling with graph size.}
To study how each centrality algorithm scales with the number of vertices, we generated Erd\H{o}s--R\'enyi random graphs $G(n,p)$ for increasing $n$ while keeping the edge probability fixed at $p = 0.1$. For each $n$, we generated a fresh graph, ran all four centrality algorithms, and recorded their running times. The resulting size vs. time plot is shown in Figure~\ref{fig:size-time-centrality}.

\begin{figure}[h]
    \centering
    \includegraphics[width=0.7\linewidth]{./plots/centrality/images/size_vs_time.png}
    \caption{Running time of degree, harmonic closeness, betweenness (Brandes), and PageRank as a function of graph size $n$ for Erd\H{o}s--R\'enyi graphs $G(n,0.1)$.}
    \label{fig:size-time-centrality}
\end{figure}

Empirically, we observe the following patterns:
\begin{itemize}
    \item \textbf{Degree centrality} is consistently the fastest algorithm. The measured running time grows roughly linearly with $n$, in line with the theoretical complexity $\Theta(n + m)$ for adjacency-list graphs.
    \item \textbf{Harmonic closeness} and \textbf{betweenness (Brandes)} both exhibit a much steeper increase in running time as $n$ grows. This matches the theoretical cost of performing a breadth-first search (BFS) from every node, $\Theta(n(n + m))$.
    \item \textbf{PageRank} lies between degree centrality and the BFS-based algorithms. It is implemented via power iteration with a fixed maximum number of iterations and a convergence tolerance, so its running time scales roughly linearly with $m$ per iteration.
\end{itemize}

Overall, the size-vs-time experiment matches the theoretical expectations: simple local measures (degree) are essentially linear, global shortest-path based measures (harmonic closeness, betweenness) are the most expensive, and PageRank occupies a middle ground.

\paragraph{Scaling with graph density.}
To study how each centrality algorithm behaves as the graph becomes denser, we fixed the number of vertices at $n = 100$ and varied the edge probability $p$ in the Erd\H{o}s--R\'enyi model $G(n,p)$. For each value of $p$, a fresh random graph was generated and the running times of all four centrality algorithms were recorded. The resulting density--runtime plot is shown in Figure~\ref{fig:density-time-centrality}.

\begin{figure}[h]
    \centering
    \includegraphics[width=0.7\linewidth]{./plots/centrality/images/density_vs_time.png}
    \caption{Running time of degree, harmonic closeness, betweenness (Brandes), and PageRank as a function of edge probability $p$ for Erd\H{o}s--R\'enyi graphs $G(100,p)$.}
    \label{fig:density-time-centrality}
\end{figure}

Empirically, the following patterns are observed:
\begin{itemize}
    \item \textbf{Degree centrality} remains almost constant across all values of $p$, matching the theoretical complexity $\Theta(n + m)$ when $n$ is fixed.
    \item \textbf{Harmonic closeness} and \textbf{betweenness (Brandes)} both show a steep increase as $p$ grows because both algorithms run BFS from every vertex, yielding $\Theta(n(n+m))$ and $\Theta(nm)$ time respectively.
    \item \textbf{PageRank} displays a non-monotonic trend: sparse graphs with many dangling nodes run slower, mid-density graphs converge faster, and very dense graphs become slower again as the per-iteration cost approaches $\Theta(n+m)$.
\end{itemize}

\paragraph{Qualitative comparison using a showcase graph.}
We also constructed a small, intentionally structured showcase graph to illustrate the differing behaviours of the four measures. Two dense communities, a single bridge, a hub with leaf attachments, and a peripheral chain allow each algorithm to highlight distinct structures. Figure~\ref{fig:showcase-heatmaps} overlays the four heatmaps using a fixed layout so colour intensity directly reflects the normalised centrality score of each node.

\begin{figure}[h]
    \centering
    \includegraphics[width=\linewidth]{./plots/centrality/images/centrality_heatmap_grid.png}
    \caption{Heatmaps of degree, harmonic closeness, betweenness, and PageRank on the manually constructed showcase graph.}
    \label{fig:showcase-heatmaps}
\end{figure}

\paragraph{Summary of findings.}
\begin{itemize}
    \item Degree centrality is extremely fast and captures local connectivity, making it suitable for very large graphs when only local importance is needed.
    \item Harmonic closeness and betweenness centrality provide richer global information but are computationally expensive, scaling roughly like $\Theta(nm)$ on unweighted graphs.
    \item PageRank provides a more global perspective than degree centrality while remaining cheaper than BFS-from-every-node approaches.
\end{itemize}
These empirical results align with the theoretical analyses and clarify the practical trade-offs between different notions of centrality in networks.
